\begin{center}
	\section*{\centering ABSTRACT}\addcontentsline{toc}{section}{ABSTRACT}
\end{center}
\hspace{1.5cm}
Wheeled mobile robotic path planning is one of the main problem in the robot navigation task. Path planning allows the robot to navigate inside surrounding environment from point A to point B while avoiding the obstacle such as wall, furniture, human, -etc. To plan the path that it needs to take in the environment, the robot needs a right quantity of information of its surrounding and algorithm that will determine the optimal path for the navigation. This thesis presents an algorithm of path planning, control, and localization for the robot. An experiment is conducted inside the simulation environment using Gazebo and ROS software. The robot kinematic and dynamics of differential drive mobile are derived. The backstepping controller is used to control the robot motion. To crate a map of the environment (Occupancy Grid Map), we apply Simultaneous Localization and Mapping (SLAM) method called HectorSLAM. A* path planning algorithm is used to find the path for the robot to move. Extended Kalman Filter is used with the kinematic model to localize the robot position in the map. We use three simulated sensors such as: inertial measurement unit (IMU), wheel encoder, and light detection and ranging (Lidar). The noise of each sensor is assumed to be white Gaussian. In the experiment, the robot performs in two different cases: "Map1" and "Map2". The result shows the collected data from SLAM, Path, and Control of the robot.\par